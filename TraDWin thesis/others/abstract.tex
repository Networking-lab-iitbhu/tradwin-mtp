\newpage
\chapter*{\centering Abstract}
{
In the current era marked by rapid urbanization and population growth, cities worldwide face new challenges, chief among them being the effective management of traffic congestion and the optimization of transportation infrastructure. With reliable real-time data available through the emerging potential of smart cities powered by 5G and the Internet of Things (IoT), the use of sophisticated cyberphysical systems like Digital Twins is becoming increasingly viable. Such systems, powered by the ready availability of large amounts of data and machine learning methods, can prove to be increasingly effective at modeling the dynamics of urban traffic flow in real-time.

In this paper, we present \modelname, which is a Traffic Digital Twin (TDT) designed to seamlessly aggregate streaming traffic data from a myriad of sources, including traffic sensors, cameras, GPS devices, and more. Through the integration of physics-informed deep machine learning models, our model serves as an alternative to traditional empirical methods of slow and resource-intensive large-scale micro-simulations. By incorporating road network information, traffic volumes at different points, and other semantic information such as road characteristics and weather, the interactive TDT is designed and tested for three crucial aspects related to urban traffic state estimation: (i) prediction of future traffic states (ii) imputation of traffic state information on missing links, and (iii) traffic re-assignment on changing map topology, which can be useful for scenarios such as road closures or new construction planning. We test and validate our model using traffic volume data across Dublin city obtained from the SCATS traffic management system and also through SUMO-based micro-simulations of relevant scenarios.
}