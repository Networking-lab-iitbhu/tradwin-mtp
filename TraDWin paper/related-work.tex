%!TEX root=./paper.tex
\section{Related Work}\label{sec:related-works}

In this section, we review the existing literature on traffic modeling, focusing on current spatio-temporal modeling methods. We compare these approaches with our framework, addressing recent work relevant to each of the specific tasks our model aims to perform: prediction, imputation, and prediction in response to changing topology, the latter being a distinctive feature of our work.

Autoregressive Integrated Moving Average (ARIMA) \cite{arima} and its variants, such as time-oriented ARIMA \cite{time_arima} and Seasonal ARIMA \cite{sarima}, are widely used traditional algorithms for prediction and forecasting tasks. ARIMA operates on time series data and is frequently combined with other algorithms to incorporate spatial features. For instance, C. Xu et al. \cite{Xu2016} integrated ARIMA with a genetic algorithm to estimate future traffic flow on roads. While ARIMA models are proficient in capturing linear time series data, the addition of genetic algorithms enhances their ability to extract features from nonlinear historical data. This integration demonstrates the versatility and adaptability of ARIMA-based approaches in addressing complex prediction challenges. Consequently, we also compare our model's performance with that of the ARIMA model. But one major problem this method suffers from is the lack of spatial information about the road network, which is crucial for traffic prediction tasks.

Other methods, such as Temporal Regularized Matrix Factorization (TRMF) \cite{trmf}, Bayesian Temporal Regularized Matrix Factorization (BTRMF) \cite{btrmf}, and Bayesian Temporal Matrix Factorization (BTMF) \cite{btrmf}, have also been explored in the context of traffic prediction. These methods extend the principles of matrix and tensor factorization to capture temporal dependencies and spatial correlations in high-dimensional traffic data.

TRMF, introduced by Lai et al. (2017), extends matrix factorization to tensor data structures, enabling the modeling of multi-dimensional traffic flow data with enhanced flexibility. Similarly, BTRMF, proposed by Liu et al. (2019), integrates Bayesian inference into tensor factorization to provide probabilistic predictions and uncertainty quantification in traffic forecasting tasks. BTMF and Bayesian Temporal Tensor Factorization (BTTF) further extend the Bayesian framework to matrix and tensor factorization, respectively, offering robust approaches for modeling temporal dynamics and spatial interactions in traffic networks while accounting for uncertainties in the prediction process.

These methods contribute to the diverse landscape of predictive models for traffic forecasting, each offering unique advantages and insights for accurately addressing the challenges of traffic flow prediction. But again, these methods are limited in their ability by the lack of spatial information about the road network, as was the case with ARIMA.
Next, we explore GNNs which is a new class of models that try to address this limitation.

Graph Neural Networks (GNNs) \cite{gnn} have emerged as a prominent tool for spatiotemporal modeling, demonstrating effectiveness in various applications such as traffic forecasting and imputation. Studies by Li et al. (2018) \cite{li2018gnn} and Zhang et al. (2019) \cite{wavenet} have highlighted their utility in accurately predicting traffic patterns and filling in missing data in traffic volume time series. GNNs iteratively update node representations based on local neighborhood information, allowing them to capture complex spatial and temporal dependencies within graph-structured data. 

However, a limitation of traditional GNNs is their reliance on a static network topology during training, which poses challenges when adapting them to dynamic graphs. This limitation is particularly relevant for tasks involving changes in the graph structure, such as predicting traffic flow when adding or removing edges, a crucial aspect of our research focus.

Physics-informed deep learning (PIDL) \cite{raissi2017physics} is an emerging methodology wherein a neural network is trained to perform learning tasks while adhering to the principles of physics, as defined by physics-based constraint equations and general nonlinear partial differential equations. By embedding the fundamental laws of physics as biases during training, the model is not required to independently discover these dependencies. This approach enhances the data efficiency of the resultant neural network.

For traffic modeling, physics-informed deep learning (PIDL) provides an effective middle ground between purely data-driven models and purely model-driven methods. Purely model-driven approaches use mathematical models and prior knowledge of traffic dynamics to estimate future states, assuming the model accurately represents real-world traffic. However, this assumption often fails to capture the intricate dynamics of real-world traffic, and multiple equally viable models can exist for the same task, making generalization challenging. Examples of model-driven traffic approaches include the Lighthill-Whitham-Richards (LWR) \cite{lwr} model and the Cell Transmission Model (CTM) \cite{ctm}. 

In contrast, pure deep learning approaches require large amounts of data to learn generalized relationships, as they lack pre-existing information on physical relations and constraints. PIDL combines both approaches by incorporating physics-based biases into the deep learning model through a parameter \(\lambda\), while still allowing the model enough freedom to learn more granular relationships in traffic dynamics. This integration enables the model to leverage the strengths of both methodologies, enhancing its effectiveness in traffic modeling.

In summary, while existing methods have made significant strides in traffic prediction and imputation tasks, they often fall short in integrating spatial information about the road network, which is vital for capturing complex traffic dynamics. Traditional approaches like ARIMA and its variants, though effective in linear time series prediction, lack spatial awareness. Advanced matrix factorization techniques such as TRMF, BTRMF, and BTMF offer improved temporal modeling but still struggle with spatial information integration. GNNs have shown promise by leveraging graph structures to model spatiotemporal dependencies; however, their inability to adapt to dynamic network topologies limits their applicability in scenarios involving changing road networks. Physics-informed deep learning (PIDL) provides a balanced approach by combining data-driven and model-driven methods, yet existing models have not fully explored their application to traffic dynamics in the context of changing network structures. Our research addresses this gap by focusing on predicting traffic flow in dynamic road networks, a novel problem area previously tackled primarily through simulation-based models like Vissim\cite{vissim} and SUMO\cite{sumo}.