%!TEX root=./paper.tex
\section{Conclusion}\label{sec:conclusion}
In conclusion, we propose an end-to-end framework for an interactive traffic digital twin, including a model. We evaluate and compare our model's performance on the real-world Dublin SCATS dataset, and our experiments show that the model produces reasonable results compared to other contemporary approaches and techniques in different scenarios. We believe that such a system will be increasingly useful for managing and guiding the decision-making processes related to traffic infrastructure and planning, both by governments and private entities.

It is worth noting that the core architecture of our TDT model is the same for all three underlying tasks. In our approach, we treat each task as a separate problem for the model to be trained upon. Given that multi-task learning for models, where the model can better generalize by learning several different tasks, with output controlled by the parameters of the input itself, has shown promising results, we believe it will be worthwhile to investigate the application of that approach to our model.
