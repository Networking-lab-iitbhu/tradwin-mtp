\begin{abstract}

In the contemporary world characterized by rapid urbanization and burgeoning populations, cities worldwide confront pressing challenges, notably the effective management of traffic congestion and the optimization of transportation infrastructure. The advent of smart cities facilitated by 5G and the Internet of Things (IoT) has unlocked unprecedented opportunities to harness real-time data. This technological advancement has spurred the adoption of sophisticated cyberphysical systems such as Digital Twins, which are increasingly recognized for their potential in modeling urban traffic dynamics with precision.

This study introduces \name, an end-to-end framework for a Traffic Digital Twin designed to seamlessly aggregate real-time traffic data sourced from diverse sensors, cameras, GPS devices, and analogous instruments. By leveraging physics-informed deep machine learning models, our framework represents a departure from traditional empirical approaches reliant on laborious and resource-intensive micro-simulations. \name\ integrates comprehensive datasets encompassing road network configurations, traffic volumes at distinct junctures, and contextual parameters including road characteristics and meteorological conditions. Trained separately for each, it is able to address three pivotal aspects of urban traffic state estimation: (i) prediction of future traffic conditions, (ii) imputation of traffic state data for segments with missing information, and (iii) traffic assignments in response to evolving map topologies, crucial for scenarios such as road closures or urban development initiatives. The framework ensures compatibility with the different possible data sources you may have as well as re-learning based on new emerging patterns. Finally, we evaluate our proposed model using traffic volume data obtained from Dublin city's SCATS traffic management system, alongside simulation-based assessments using the SUMO platform to replicate pertinent scenarios.
\end{abstract}

\begin{IEEEkeywords}
Digital Twins, Traffic, Traffic state estimation, Traffic imputation, Traffic prediction, Urban Planning, Physics Informed Model
\end{IEEEkeywords}